% pdflatex -shell-escape template_cdc.tex

\documentclass[a4paper,oneside]{article}

\usepackage[frenchb]{babel}
\usepackage[utf8]{inputenc}
\usepackage[T1]{fontenc}
\usepackage{graphicx}
\usepackage{amssymb} 
\usepackage{amsmath}
\usepackage{hyperref}
\usepackage{fullpage}
\usepackage{epstopdf}


%%%%%%%%%%%%%%%%%%%%%%%%%

\newcommand{\mytitle}{Projet Puissance X - Cahier des charges}
\title{\mytitle } 

%%%%%%%%%%%%%%%%%%%%%%%%%

\makeatletter

\usepackage{fancyhdr}
\pagestyle{fancyplain}
\fancyhf{}
\renewcommand{\headrulewidth}{0pt}
\renewcommand{\footrulewidth}{0.5pt}
\lfoot{\mytitle}
\cfoot{\@date}
\rfoot{page \thepage / \pageref{fin}}

\author{Benjamin cleton, Clement ansel et Matthieu laniesse}

\date{29 mai 2017}


%%%%%%%%%%%%%%%%%%%%%%%%%


\begin{document}

\maketitle

\thispagestyle{fancyplain}


%%%%%%%%%%%%%%%%%%%%%%%%%

\section{Renseignements}

\paragraph{Nom du projet :}
Puissance X

\paragraph{Objet :}
Développement d'un jeu en tour par tour contre un ordinateur

\paragraph{Maître d'ouvrage :}
Benjamin cleton, Clement ansel, Matthieu laniesse

\paragraph{Maître d'oeuvre : }
Benjamin cleton, Clement ansel, Matthieu laniesse

\paragraph{Date de début :}
29 mai 2017

\paragraph{Date de fin :}
16 juin 2017


%%%%%%%%%%%%%%%%%%%%%%%%%

\newpage

\section{Définition du besoin}

\paragraph{Contexte général\\}
Benjamin, Clément et Matthieu sont 3 étudiants qui ont pour projet de créer un jeu avec une intelligence artificielle dans le cadre de leurs études.
Nostalgique de leur enfance, l'idée d'un puissance X leur est venu comme une évidence.


\paragraph{Besoins et priorités\\}
Le besoin principal est de pouvoir jouer au puissance X de plusieurs manières :
\begin{itemize}
	\item Ordinateur versus ordinateur
	\item Joueur versus ordinateur
\end{itemize}
Les priorités sont que le jeu puisse être jouer par le plus grand nombres, qu'il soit divertissant et enfin que l'interface utilisateur soit  simple et intuitive.


%%%%%%%%%%%%%%%%%%%%%%%%%

\newpage

\section{Spécifications}

\begin{itemize}
    \item Jeu fonctionnant sur tous les ordinateurs vendu dans le commerce.
    \item fonctionnalités du Puissance X :
        \begin{itemize}
        	\item Choisir le nombre de pions alignés permettant la victoire
        	\item Choisir la taille de la grille
            \item Placer des jetons dans une grille
            \item Afficher les scores
            \item Choisir un Pseudonyme 
            \item Choisir un mode de jeu
            \item Choisir un degré de difficulté
            \item Possibilité de revenir un coup en arrière
            \item Implémentation d'un son après un coup joué
        \end{itemize}
    \item interface utilisateur :
        \begin{itemize}
            \item Affichage de la grille de jeu en deux dimensions
            \item Affichage des Pseudonymes et des joueurs
            \item Affichage des boutons de réglages
            \item Affichage d'un temps de réflexion pour jouer
            \item Affichage d'un menu après fin d'une partie
        \end{itemize}
    \item performances demandées :
        \begin{itemize}
            \item temps de réponse de l'ordinateur réglé en fonction de sa difficulté
            \item temps de réponse des actions inférieurs à 500ms
        \end{itemize}

\end{itemize}


%%%%%%%%%%%%%%%%%%%%%%%%%

\newpage

\appendix

\section{Livrables}

\begin{itemize}
    \item Jeu déployé sur tout les ordinateurs vendu dans le commerce
    \item code source documenté sous licence WTFPL v2 //
    \item manuel d'installation et de configuration //
    \item manuel d'utilisation //
\end{itemize}







\section{Maquettes}

\paragraph{Page du Menu Principal}

~\\

\includegraphics[width=9cm]{image2.png}

\paragraph{Page de Jeu}

~\\

\includegraphics[width=13cm]{image1.png}






\section{Planning prévisionnel}

\includegraphics[width=13cm]{cdc_gantt.png}


\section{...}


%%%%%%%%%%%%%%%%%%%%%%%%%

\label{fin}

\end{document}


